\documentclass{pramana}

%%
%%download pramana.cls and save it in the folder of  your source file
%%

%%suggested packages to be included
\usepackage{graphicx,amsmath,bm}

%%The following packages are included with the class file.
%%Please download if these packages are not included
%%in your local TeX distribution 
%%txfonts,balance,textcase,float
%%

\begin{document}

%%paper title
%%For line breaks \\ can be used within title 
\title{A Primer on Machine Learning for Physical Sciences}


%%author names are separated by comma (,) 
%%use \and before the last author name 
%%\textsuperscript{number} is used for affiliation
%%use a * along with the number separated by comma
%% for the  author for correspondence

\author{Tamil Arasan Bakthavatchalam\textsuperscript{1}, Selva Kumar\textsuperscript{1}\and Suriyadeepan Ramamoorthy\textsuperscript{2,*}}
\affilOne{\textsuperscript{1} Department of P, University X\\}
\affilTwo{\textsuperscript{2} Department of Q, University Z}

%%escape two column mode for title, affiliation and abstract
%%by giving \twocolumn command as shown

\twocolumn[{

\maketitle

%%include \corres to print the corresponding author Email id
\corres{abc@xyz.com}

%%include \msinfo for
%%manuscript information such as
%%received, revised and accepted dates
%%
\msinfo{1 January 2015}{1 January 2015}{1 January 2015}

%%abstract
\begin{abstract}
Machine Learning has shown substantial impact on Scientific Computing in recent years. The adaptation of ML techniques to deal with various research problems in Physical Science have gained ground as an alternative to the existing Numerical Schemes. In this paper, we would like to introduce the reader to a basic Neural Networks(NN) that can solve Ordinary Differential Equations and Partial Differential Equations. In Particular we choose First and Second Order equations to illustrate the proficiency of Neural Network over traditional numerical techniques. This paper will be helpful for any Graduate and Undergraduate students and also as an introductory material to any new researcher who wants to apply Machine Learning techniques to their problem of interest. 
\end{abstract}

%%insert keywords separated by comma using \keywords{words}
\keywords{keyword1, keyword2, keyword3 etc.}


%%include \pacs{number} to print the PACS number
\pacs{12.60.Jv; 12.10.Dm; 98.80.Cq; 11.30.Hv}

}]
%%close the twocolumn escape here

%%include \doinum{number}for the DOI number in the header
%%include \volnum{number} for the volume number in the header
%%include \year{yyyy} for  year of publication in the header
%%include \pgrange{num--num} page range of article in the header
%%include \artcitid{num} for the article citation id
%%include \lp to print last page of the article
%%include \setcounter{page}{pagenum} for the exact starting page of the article

\doinum{12.3456/s78910-011-012-3}
\artcitid{\#\#\#\#}
\volnum{123}
\year{2016}
\pgrange{23--25}
\setcounter{page}{23}
\lp{25}



\section{Introduction}
The field of Machine learning had brought significant changes in the development of every field that we come across. Importantly the acceleration of COVID vaccine were made possible by using appropriate ML techniques, In physical science the application Neural Network had been in practise for last 2 decades and there had been many applications of NN in research problems such as . 

The dynamics of any dynamical system can be 
Differential equations plays an important role in the development of any modern science such as in Physics(Heat Transfer), Biology(Exponential Growth and Decay), Economics(Banking Interest) and many more applications can be found in literature. In general Differential Equation is an equation which maps the functions values with its derivatives values. It can be classified into two large groups namely Linear Equations and Nonlinear Equations and an ordinary differential equation is that a function with one variable whereas partial differential equation does depend on several variables. Most of the Nonlinear differential equations are difficult to solve using analytical techniques which leads one to rely on numerical techniques. There are several numerical schemes which can be used to solve differential equations such as Finite Difference Scheme,Finite Element Method, Crank Nicholson Scheme, etc. These methods do depend on numerical differentiation which relies on the mesh which is in general known as grids. But Machine Learning techniques adopts Automatic Differentiation which is a mesh-free approach.
\subsection{Subsection heading}
Subsection text here. 


\section{A new Section}
Some more  text here.  

An equation
\begin{equation}
E=mc^2
\end{equation}

%%Use table environment for a table in one column

\begin{table}[htb]
\caption{Table fitting in a single column}\label{tableExample}
\begin{tabular}{|l|cccc|r|}
\hline
one& two &three&four&five&six\\
1&2&3&4&5&6\\
aaa&bbbb& ccccc&dddd&eeeee&ffffff\\
\hline
\end{tabular}
%%use \tablenotes{footnote} to get the table foot note
\tablenotes{Sample table footnote}
\end{table}


%%Use table* environment to get the table spanning both the columns

%\begin{table*}[htb]
%\caption{Caption text here}\label{secondTable}
%\begin{tabular}{|l|cccccccc|r|}
%\hline
%\textbf{head1}&\multicolumn{8}{|c|}{\textbf{head2}}&\textbf{head3}\\
%\hline
%one& two &three&four&five&six&seven&eight&nine&ten\\
%1&2&3&4&5&6&7&8&9&10\\
%aaa&bbbb&cccc&ddddd&eee&ffff&ggggg&hhhhhhhh&iiii&jjjjjj\\
%\hline
%\end{tabular}
%\tablenotes{table footnote here}
%*table spanning both the columns
%\end{table*}


%%An example of a figure

%\begin{figure}[!t]
%\centering{
%\includegraphics[width=.8\columnwidth]{fig1.eps}
%}
%\caption{caption goes here}\label{figOne}
%\end{figure}

%%An example of a double column figure
%%Use figure* environment

%\begin{figure*}
%\centering\includegraphics[height=.15\textheight]{fig2.eps}
%\caption{caption spanning two columns}
%\centering\includegraphics[height=.25\textheight]{fig3.eps}
%\caption{caption here}
%\end{figure*}


\section{Conclusion}
Conclusion here.


%%Appendix

\appendix

\section{An appendix section}
Text goes here.
\begin{equation}
x=a+b+c
\end{equation}
\section{Another appendix section}
Text goes here.
\begin{equation}
y^2=ax+b+c
\end{equation}
%%Use section* for acknowledgements
\section*{Acknowledgement}
Acknowledgements

%%use \balance somewhere in the left column of the last page to balance the two columns in the end page

%%References section
\begin{thebibliography}{99} 
\bibitem{latexcompanion} 
Michel Goossens, Frank Mittelbach, and Alexander Samarin. 
\newblock {\em The \LaTeX\ Companion}. 
Addison-Wesley, Reading, Massachusetts, 1993.
\end{thebibliography}

\end{document}
